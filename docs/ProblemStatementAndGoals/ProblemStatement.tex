\documentclass{article}

\usepackage{tabularx}
\usepackage{booktabs}

\title{Problem Statement and Goals\\Equivariant Sensor Fusion}

\author{Alaap Grandhi}

\date{}

%% Comments

\usepackage{color}

\newif\ifcomments\commentstrue %displays comments
%\newif\ifcomments\commentsfalse %so that comments do not display

\ifcomments
\newcommand{\authornote}[3]{\textcolor{#1}{[#3 ---#2]}}
\newcommand{\todo}[1]{\textcolor{red}{[TODO: #1]}}
\else
\newcommand{\authornote}[3]{}
\newcommand{\todo}[1]{}
\fi

\newcommand{\wss}[1]{\authornote{blue}{SS}{#1}} 
\newcommand{\plt}[1]{\authornote{magenta}{TPLT}{#1}} %For explanation of the template
\newcommand{\an}[1]{\authornote{cyan}{Author}{#1}}

%% Common Parts

\newcommand{\progname}{ProgName} % PUT YOUR PROGRAM NAME HERE
\newcommand{\authname}{Team \#, Team Name
\\ Student 1 name
\\ Student 2 name
\\ Student 3 name
\\ Student 4 name} % AUTHOR NAMES                  

\usepackage{hyperref}
    \hypersetup{colorlinks=true, linkcolor=blue, citecolor=blue, filecolor=blue,
                urlcolor=blue, unicode=false}
    \urlstyle{same}
                                


\begin{document}

\maketitle

\begin{table}[hp]
\caption{Revision History} \label{TblRevisionHistory}
\begin{tabularx}{\textwidth}{llX}
\toprule
\textbf{Date} & \textbf{Developer(s)} & \textbf{Change}\\
\midrule
Jan. 21 & Alaap & Initial Commit\\
\midrule
Apr. 16 & Alaap & Updated to incorporate feedback \\
\bottomrule
\end{tabularx}
\end{table}

\section{Problem Statement}

Over the past few years, we have seen more and more autonomous vehicles being allowed onto the road. 
With this increase, the need for perception methods that can effectively use readings from sensor 
modalities like Camera and LiDAR to inform these vehicles of their surroundings has similarly increased.

\subsection{Problem} 
While sensor modalities like LiDAR and Camera can capture meaningful information about road conditions,
they are inherently prone to failure. LiDAR sensors can fail in scattering media like snow while cameras
are easily occluded. Thus, learnt methods combining the readings from these modalities have become the 
state-of-the-art for this task due to their ability to overcome individual sensor modality failures.
Still, these techniques suffer from local misalignment between the features obtained for these modalities. 
This motivates the need for new approaches that can robustly combine readings from these modalities.  

\subsection{Inputs and Outputs}
For the purpose of this document, the process by which the given network/model is trained is out of scope. 
Given this, the inputs in this problem would be a set of camera images taken from different poses (multi-view) alongside a 
pointcloud obtained from a lidar sensor. Subsequently, the outputs would be a set of labelled bounding box predictions 
corresponding to pedestrians and vehicles observed in the robot's surroundings. 

\subsection{Stakeholders}
\subsubsection{Academic Stakeholders}
Academic stakeholders for this task would primarily include Computer Vision and Robot Path Planning researchers
in the field of Autonomous Driving. With the sheer volume of research being conducted in this field using public datasets,
this work would aim to be an easily integratable and easily adaptable approach to alignment-aware sensor fusion for autonomous
driving. Thus, it could serve as a basis for future research in the field by academic institutions. 

\subsubsection{Industrial Stakeholders}
Industrial stakeholders for this task would primarily include companies like Waabi, Waymo and Tesla that produce
autonomous vehicles. 

\subsubsection{Other Stakeholders}
Aside from the downstream users of this software, the ARCO lab (the lab I am a part of) and my research supervisor (Dr. Matthew Giamou) would also be stakeholders for this project.
I am aiming to turn my work into a conference paper sometime over the summer and I plan on putting his name on the paper as well.
As such, this research will hopefully expand the reach of the ARCO lab as a whole.  

\subsection{Environment}
\subsubsection{Software}
The software should be compatible with any up-to-date Linux operating system. At present, I am not planning to support 
all operating systems as the library that I am building my code on top of (OpenPCDet) is not compatible with all operating systems.


\subsubsection{Hardware}
Given that the solution space will revolve around learned methods, any computer possessing a machine-learning capable
GPU with at least 12GB vram will be suitable for training and inference of models. However, this vram requirement would
be significantly lower for inference-only settings using pre-trained models.

\section{Goals}
\begin{enumerate}
    \item Given a set of camera images $\{C_i\}$ and a lidar pointcloud $P$ obtained from an autonomous vehicle at a 
    given point in time, determine the set of bounding boxes for dynamic entities in its surroundings $\{B_i\}$ 
    \item Given a common standard training dataset (i.e. Waymo, NuScenes, Kitti), the trained model should achieve comparable 
    accuracy (mAP) to existing state-of-the-art methods
    
\end{enumerate}
\section{Stretch Goals}
\begin{enumerate}
    \item In the presence of disturbances in one of the two given modalities, the trained model should achieve comparable 
    accuracy (mAP) to a network solely trained on and processing the undisturbed modality.
    \item Given a set of disturbed camera images $\{C_{d,i}\}$ and a disturbed lidar pointcloud $P_d$, achieve 
    better accuracy (mAP) than a network solely trained on either modality.
    \item Produce a range of architectures that can each meet the aforementioned requirements and thus be chosen from based
    on the specific dataset used. 
\end{enumerate}
\section{Challenge Level and Extras}

This project would be considered an advanced research project as it will include the development
of newer state-of-the-art perception architectures (a task that typically can lead to published paper).
Additionally, although beyond the scope of this document, the approaches considered will utilize group
theoretic principles and Equivariant architectures to improve alignment between sensor modalities. This 
would in turn introduce the need for custom layers (rather than just combining a set of off-the-shelf components).

\newpage{}

\end{document}