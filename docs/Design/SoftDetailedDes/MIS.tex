\documentclass[12pt, titlepage]{article}

\usepackage{amsmath, mathtools}

\usepackage[round]{natbib}
\usepackage{amsfonts}
\usepackage{amssymb}
\usepackage{graphicx}
\usepackage{colortbl}
\usepackage{xr}
\usepackage{hyperref}
\usepackage{longtable}
\usepackage{xfrac}
\usepackage{tabularx}
\usepackage{float}
\usepackage{siunitx}
\usepackage{booktabs}
\usepackage{multirow}
\usepackage[section]{placeins}
\usepackage{caption}
\usepackage{fullpage}

\hypersetup{
bookmarks=true,     % show bookmarks bar?
colorlinks=true,       % false: boxed links; true: colored links
linkcolor=red,          % color of internal links (change box color with linkbordercolor)
citecolor=blue,      % color of links to bibliography
filecolor=magenta,  % color of file links
urlcolor=cyan          % color of external links
}

\usepackage{array}

\externaldocument{../../SRS/SRS}

\input{../../Comments}
\input{../../Common}

\begin{document}

\title{Module Interface Specification for \progname{}}

\author{\authname}

\date{\today}

\maketitle

\pagenumbering{roman}

\section{Revision History}

\begin{tabularx}{\textwidth}{p{3cm}p{2cm}X}
\toprule {\bf Date} & {\bf Version} & {\bf Notes}\\
\midrule
Date 1 & 1.0 & Notes\\
Date 2 & 1.1 & Notes\\
\bottomrule
\end{tabularx}

~\newpage

\section{Symbols, Abbreviations and Acronyms}

See SRS Documentation at \wss{give url}

\wss{Also add any additional symbols, abbreviations or acronyms}

\newpage

\tableofcontents

\newpage

\pagenumbering{arabic}

\section{Introduction}

The following document details the Module Interface Specifications for
\wss{Fill in your project name and description}

Complementary documents include the System Requirement Specifications
and Module Guide.  The full documentation and implementation can be
found at \url{...}.  \wss{provide the url for your repo}

\section{Notation}

\wss{You should describe your notation.  You can use what is below as
  a starting point.}

The structure of the MIS for modules comes from \citet{HoffmanAndStrooper1995},
with the addition that template modules have been adapted from
\cite{GhezziEtAl2003}.  The mathematical notation comes from Chapter 3 of
\citet{HoffmanAndStrooper1995}.  For instance, the symbol := is used for a
multiple assignment statement and conditional rules follow the form $(c_1
\Rightarrow r_1 | c_2 \Rightarrow r_2 | ... | c_n \Rightarrow r_n )$.

The following table summarizes the primitive data types used by \progname. 

\begin{center}
\renewcommand{\arraystretch}{1.2}
\noindent 
\begin{tabular}{l l p{7.5cm}} 
\toprule 
\textbf{Data Type} & \textbf{Notation} & \textbf{Description}\\ 
\midrule
character & char & a single symbol or digit\\
integer & $\mathbb{Z}$ & a number without a fractional component in (-$\infty$, $\infty$) \\
natural number & $\mathbb{N}$ & a number without a fractional component in [1, $\infty$) \\
real & $\mathbb{R}$ & any number in (-$\infty$, $\infty$)\\
\bottomrule
\end{tabular} 
\end{center}

\noindent
The specification of \progname \ uses some derived data types: sequences, strings, and
tuples. Sequences are lists filled with elements of the same data type. Strings
are sequences of characters. Tuples contain a list of values, potentially of
different types. In addition, \progname \ uses functions, which
are defined by the data types of their inputs and outputs. Local functions are
described by giving their type signature followed by their specification.

\section{Module Decomposition}

The following table is taken directly from the Module Guide document for this project.

\begin{table}[h!]
  \centering
  \begin{tabular}{p{0.3\textwidth} p{0.6\textwidth}}
  \toprule
  \textbf{Level 1} & \textbf{Level 2}\\
  \midrule
  
  {Hardware-Hiding Module} & ~ \\
  \midrule
  
  \multirow{7}{0.3\textwidth}{Behaviour-Hiding Module} 
  & Config Loading Module\\
  & Data Loading Module\\
  & Model Module\\
  & Checkpointing Module\\
  & Training Module\\
  & Inference Module\\
  & Loss Module\\ 
  & Evaluation Module\\
  & Optimization Module\\
  & Data Processing Module\\
  \midrule
  
  \multirow{3}{0.3\textwidth}{Software Decision Module}
  & Plotting Module\\
  & PyTorch Module\\
  & Logging Module\\
  \bottomrule
  
  \end{tabular}
  \caption{Module Hierarchy}
  \label{TblMH}
  \end{table}

\newpage
~\newpage

\section{MIS of Training Module} \label{Module} 

\subsection{Module}



\subsection{Uses}


\subsection{Syntax}



\subsubsection{Exported Constants}



\subsubsection{Exported Access Programs}

\begin{center}
\begin{tabular}{p{2cm} p{4cm} p{4cm} p{2cm}}
\hline
\textbf{Name} & \textbf{In} & \textbf{Out} & \textbf{Exceptions} \\
\hline
trainEpoch & Dataloader, Model, Loss, Optimizer & - & - \\
\hline
train & Dataloader, Model, Loss, Optimizer & - & - \\
\hline
\end{tabular}
\end{center}

\subsection{Semantics}

\subsubsection{State Variables}

N/A

\subsubsection{Environment Variables}

File System

\subsubsection{Assumptions}



\subsubsection{Access Routine Semantics}


\noindent train():
\begin{itemize}
\item transition: Saving model as a checkpoint on the file system. 
\item output: N/A
\item exception: N/A
\end{itemize}

\subsubsection{Local Functions}

Is trainEpoch a local function? It updates the model using the optimizer, dataset, and loss function.

\newpage

\section{MIS of Inference Module} \label{Module} 

\subsection{Module}



\subsection{Uses}


\subsection{Syntax}



\subsubsection{Exported Constants}



\subsubsection{Exported Access Programs}

\begin{center}
\begin{tabular}{p{2cm} p{4cm} p{4cm} p{2cm}}
\hline
\textbf{Name} & \textbf{In} & \textbf{Out} & \textbf{Exceptions} \\
\hline
infer?? & Dataloader, Model & - & - \\
\hline
\end{tabular}
\end{center}

\subsection{Semantics}

\subsubsection{State Variables}

N/A

\subsubsection{Environment Variables}

Screen

\subsubsection{Assumptions}



\subsubsection{Access Routine Semantics}


\noindent infer():
\begin{itemize}
\item transition: Displaying the predicted bounding boxes in the LiDAR pointcloud on the screen.
\item output: N/A
\item exception: N/A
\end{itemize}

\subsubsection{Local Functions}

Not yet done

\newpage

\section{MIS of Model Module} \label{Module} 

\subsection{Module}



\subsection{Uses}


\subsection{Syntax}



\subsubsection{Exported Constants}



\subsubsection{Exported Access Programs}

\begin{center}
\begin{tabular}{p{2cm} p{4cm} p{4cm} p{2cm}}
\hline
\textbf{Name} & \textbf{In} & \textbf{Out} & \textbf{Exceptions} \\
\hline
init & Configuration File & - & - \\
\hline
forward & Camera Images, LiDAR Data & Bounding Boxes & - \\
\hline
\end{tabular}
\end{center}

\subsection{Semantics}

\subsubsection{State Variables}

Model parameters,
Hyperparameters

\subsubsection{Environment Variables}

N/A

\subsubsection{Assumptions}



\subsubsection{Access Routine Semantics}


\noindent init():
\begin{itemize}
\item transition: Sets the hyperparameters according to the configuration file.
\item output: N/A
\item exception: N/A
\end{itemize}

\noindent forward():
\begin{itemize}
\item transition: N/A
\item output: Outputs predicted bounding boxes for the input Camera Images and LiDAR Pointcloud.
\item exception: N/A
\end{itemize}

\subsubsection{Local Functions}

Not yet done

\newpage

\section{MIS of Loss Module} \label{Module} 

\subsection{Module}



\subsection{Uses}


\subsection{Syntax}



\subsubsection{Exported Constants}



\subsubsection{Exported Access Programs}

\begin{center}
\begin{tabular}{p{2cm} p{4cm} p{4cm} p{2cm}}
\hline
\textbf{Name} & \textbf{In} & \textbf{Out} & \textbf{Exceptions} \\
\hline
init & Configuration File & - & - \\
\hline
forward & Predicted Bounding Boxes, Ground Truth Bounding Boxes & Loss "Value" & - \\
\hline
\end{tabular}
\end{center}

\subsection{Semantics}

\subsubsection{State Variables}

Hyperparameters

\subsubsection{Environment Variables}

N/A

\subsubsection{Assumptions}



\subsubsection{Access Routine Semantics}


\noindent init():
\begin{itemize}
\item transition: Sets the hyperparameters according to the configuration file.
\item output: N/A
\item exception: N/A
\end{itemize}

\noindent forward():
\begin{itemize}
\item transition: N/A
\item output: Outputs the loss between the predicted and ground truth bounding boxes.
\item exception: N/A
\end{itemize}

\subsubsection{Local Functions}

Not yet done

\newpage

\section{MIS of Data Loading Module} \label{Module} 

\subsection{Module}



\subsection{Uses}


\subsection{Syntax}



\subsubsection{Exported Constants}



\subsubsection{Exported Access Programs}

\begin{center}
\begin{tabular}{p{2cm} p{4cm} p{4cm} p{2cm}}
\hline
\textbf{Name} & \textbf{In} & \textbf{Out} & \textbf{Exceptions} \\
\hline
init & Configuration File & - & - \\
\hline
\end{tabular}
\end{center}

Iterator stuff?

\subsection{Semantics}

\subsubsection{State Variables}

Dataset,
Dataloader

\subsubsection{Environment Variables}

N/A

\subsubsection{Assumptions}



\subsubsection{Access Routine Semantics}


\noindent init():
\begin{itemize}
\item transition: Sets up the dataset and the data loader using the configuration.
\item output: N/A
\item exception: N/A
\end{itemize}

\subsubsection{Local Functions}

Not yet done

\newpage

\bibliographystyle {plainnat}
\bibliography {../../../refs/References}

\newpage

\section{Appendix} \label{Appendix}

\wss{Extra information if required}

\newpage{}

\section*{Appendix --- Reflection}

\wss{Not required for CAS 741 projects}

The information in this section will be used to evaluate the team members on the
graduate attribute of Problem Analysis and Design.

\input{../../Reflection.tex}

\begin{enumerate}
  \item What went well while writing this deliverable? 
  \item What pain points did you experience during this deliverable, and how
    did you resolve them?
  \item Which of your design decisions stemmed from speaking to your client(s)
  or a proxy (e.g. your peers, stakeholders, potential users)? For those that
  were not, why, and where did they come from?
  \item While creating the design doc, what parts of your other documents (e.g.
  requirements, hazard analysis, etc), it any, needed to be changed, and why?
  \item What are the limitations of your solution?  Put another way, given
  unlimited resources, what could you do to make the project better? (LO\_ProbSolutions)
  \item Give a brief overview of other design solutions you considered.  What
  are the benefits and tradeoffs of those other designs compared with the chosen
  design?  From all the potential options, why did you select the documented design?
  (LO\_Explores)
\end{enumerate}


\end{document}