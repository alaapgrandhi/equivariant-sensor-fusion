\documentclass[12pt, titlepage]{article}

\usepackage{booktabs}
\usepackage{tabularx}
\usepackage{hyperref}
\usepackage{float}
\hypersetup{
    colorlinks,
    citecolor=blue,
    filecolor=black,
    linkcolor=red,
    urlcolor=blue
}
\usepackage[round]{natbib}

%% Comments

\usepackage{color}

\newif\ifcomments\commentstrue %displays comments
%\newif\ifcomments\commentsfalse %so that comments do not display

\ifcomments
\newcommand{\authornote}[3]{\textcolor{#1}{[#3 ---#2]}}
\newcommand{\todo}[1]{\textcolor{red}{[TODO: #1]}}
\else
\newcommand{\authornote}[3]{}
\newcommand{\todo}[1]{}
\fi

\newcommand{\wss}[1]{\authornote{blue}{SS}{#1}} 
\newcommand{\plt}[1]{\authornote{magenta}{TPLT}{#1}} %For explanation of the template
\newcommand{\an}[1]{\authornote{cyan}{Author}{#1}}

%% Common Parts

\newcommand{\progname}{ProgName} % PUT YOUR PROGRAM NAME HERE
\newcommand{\authname}{Team \#, Team Name
\\ Student 1 name
\\ Student 2 name
\\ Student 3 name
\\ Student 4 name} % AUTHOR NAMES                  

\usepackage{hyperref}
    \hypersetup{colorlinks=true, linkcolor=blue, citecolor=blue, filecolor=blue,
                urlcolor=blue, unicode=false}
    \urlstyle{same}
                                


\newcommand{\ProjectName}{ESF }

\begin{document}

\title{System Verification and Validation Plan for \progname{}} 
\author{\authname}
\date{\today}
	
\maketitle

\pagenumbering{roman}

\section*{Revision History}

\begin{tabularx}{\textwidth}{p{3cm}p{2cm}X}
\toprule {\bf Date} & {\bf Version} & {\bf Notes}\\
\midrule
February 23 & 1.0 & Initial version of the VnV Plan\\
\bottomrule
\end{tabularx}

~\\
\wss{The intention of the VnV plan is to increase confidence in the software.
However, this does not mean listing every verification and validation technique
that has ever been devised.  The VnV plan should also be a \textbf{feasible}
plan. Execution of the plan should be possible with the time and team available.
If the full plan cannot be completed during the time available, it can either be
modified to ``fake it'', or a better solution is to add a section describing
what work has been completed and what work is still planned for the future.}

\wss{The VnV plan is typically started after the requirements stage, but before
the design stage.  This means that the sections related to unit testing cannot
initially be completed.  The sections will be filled in after the design stage
is complete.  the final version of the VnV plan should have all sections filled
in.}

\newpage

\tableofcontents

\listoftables
\wss{Remove this section if it isn't needed}

\listoffigures
\wss{Remove this section if it isn't needed}

\newpage

\section{Symbols, Abbreviations, and Acronyms}

\renewcommand{\arraystretch}{1.2}
\begin{tabular}{l l} 
  \toprule		
  \textbf{symbol} & \textbf{description}\\
  \midrule 
  T & Test\\
  \bottomrule
\end{tabular}\\

\wss{symbols, abbreviations, or acronyms --- you can simply reference the SRS
  \citep{SRS} tables, if appropriate}

\wss{Remove this section if it isn't needed}

\newpage

\pagenumbering{arabic}

This document ... \wss{provide an introductory blurb and roadmap of the
  Verification and Validation plan}

\section{General Information}

\subsection{Summary}

The \ProjectName{}project will involve constructing a library that enables users to both
train and test multi-sensor 3D object detection networks on public autonomous driving datasets. 
Expanding upon the capabilities of the OpenPCDet repository, it will serve as a foundation for 
further research into the field by presenting a method for LiDAR-Camera fusion-based object 
detection that better addresses alignment issues faced by prior methods. 

The training and inference (testing) portions of \ProjectName{}correspond to two different use-cases 
that require different considerations. During training, the library is intended to allow an experienced
machine learning and perception engineer to train a LiDAR-Camera fusion network on an autonomous driving 
dataset of their choosing. During inference, the library is intended to allow a user or a greater system 
to predict the set of dynamic objects in an autonomous vehicle's surroundings using a trained model and 
sensor input.

\subsection{Objectives}

One of the two primary objectives is to build confidence in the correctness of the software.
As a routinely used and validated open-source repository, the base functionality of the OpenPCDet
repository is considered to be error-free and thus its correctness is outside the scope of this document/project.
The installation of the software and the extensions made to it are the components for which correctness will be 
verified. 

Alongside correctness, verifying the accuracy of the solution relative to existing baselines implemented in 
OpenPCDet is the second primary objective. Since the software is intended to serve as a foundation
for future research, its relative accuracy compared to state-of-the-art baseline methods will serve 
as a metric to judge how much it contributes to research in the field.

A secondary objective is ensuring the usability of the software. Since the library will be extended and
used for research purposes, it is important to minimize the effort needed to understand and modify the code
as needed. 

\subsection{Relevant Documentation}

The other relevant design documents as well as their relations to this document are listed as follows:

\begin{itemize}
  \item SRS (\citet{SRS}): Describes the problem the software solves, the software's goals, 
  the relevant theoretical models, and the requirements for potential solutions. This document 
  thereby establishes the setting and mathematical tools necessary for the validation and verification
  methods described in this document.
  \item More documents will be added as they are completed.
\end{itemize}

\section{Plan}

\wss{Introduce this section.  You can provide a roadmap of the sections to
  come.}

\subsection{Verification and Validation Team}

\begin{table}[H]
  \centering
  \begin{tabular}{|p{3cm}|p{3cm}|p{6cm}|}
  \hline
  \textbf{Name} & \textbf{Role} & \textbf{Description} \\ \hline
  Alaap Grandhi & Author & Write the VnV plan, carry out the testing detailed in it, fill in the VnV report, and validate the software and documents against the requirements during development. \\ \hline
  Spencer Smith & Project Supervisor/Course Instructor & Review and provide feedback on the VnV plan and report. Additionally, provide insight into better testing methods. \\ \hline
  Matthew Giamou & Masters Supervisor & Provide advice on the project in general. \\ \hline
  Bo Liang & Domain Expert Reviewer & Review and provide feedback on the VnV plan and report. \\ \hline 

  \end{tabular}
\end{table}

\subsection{SRS Verification Plan}

The SRS will be reviewed in two stages. First, the Project Supervisor and the Domain
Expert Reviewer will evaluate the document using the SRS checklist from lecture (cite here).
Since they are familiar with the document structure, they will be able to provide feedback
on how well the document adheres to the template and the required structure (in addition to 
providing feedback on document content). 

Following the creation of a series of github issues corresponding 
to feedback on the document, the second stage of SRS reviewing will begin. At that point, a meeting 
will be set up with my Masters supervisor to go over the scientific principles discussed in the SRS document 
alongside the suggested changes and questions from the issues on github. Since he is familiar with 
my research he can provide additional feedback on any technical inconsistencies or ambiguities in the document.

\subsection{Design Verification Plan}

To verify the design, I have constructed the following checklist detailing the aspects of the design
that the Domain Expert Reviewer and optionally the Masters and Project Supervisors (if they have time) 
will evaluate during regular discussions:
\begin{itemize}
  \item Does the design explicitly or implicitly address all requirements described in the SRS document?
  \item Are the interfaces between modules of the design clear and in-line with the modular structure of the OpenPCDet repository?
  \item Given the overall design description, can the Domain Expert Reviewer accurately roughly predict the downstream code structure?
  \item Are any aspects of the design ambiguous?
  \item Does the design leverage existing functionality from the OpenPCDet repository where applicable?
  \item Do all of the design components contribute towards meeting a requirement?
  \begin{itemize}
    \item If not, is this due to a redundancy in the design or something missing in the SRS document?
  \end{itemize}
\end{itemize}

\subsection{Verification and Validation Plan Verification Plan}

The VnV plan will be verified in two parts corresponding to the two different 
types of tests it contains. 

Where possible, tests governed by automated test suites
like pytest will be tested using mutation testing to ensure that these tests sufficiently
cover coding mistakes that are expected to happen. The specific `mutations' to be inserted
into the code will be decided after the implementation is done. 

Manual tests and the overall structure of the VnV plan will be reviewed by the Project Supervisor
and the Domain Expert Reviewer. This will be done to both ensure that the document adheres to the
template and to ensure that the tests presented sufficiently cover the requirements described in
the SRS document.

\subsection{Implementation Verification Plan}

\wss{You should at least point to the tests listed in this document and the unit
  testing plan.}

\wss{In this section you would also give any details of any plans for static
  verification of the implementation.  Potential techniques include code
  walkthroughs, code inspection, static analyzers, etc.}

\wss{The final class presentation in CAS 741 could be used as a code
walkthrough.  There is also a possibility of using the final presentation (in
CAS741) for a partial usability survey.}

\subsection{Automated Testing and Verification Tools}

There are a few different automated tools that will be used for testing.
Since the library will be written in Python (the OpenPCDet library
is written in Python), PyFlakes will be used as an error linter to discover
simple logical errors. Linters like flake8 that enforce style convention will
not be used as the base OpenPCDet repository does not adhere to any particular
style convention (it is designed with research and quick development as the primary
objective) and it would be out of scope to refactor that. Additionally, the automated
unit tests below will be implemented in pytest and will be set to automatically run
on push using github actions. Code coverage will not be considered for this project
since by nature different components are meant to be called according to the training
configuration passed in and it would overly tedious to routinely test over all possible
input configurations.

\subsection{Software Validation Plan}



\wss{If there is any external data that can be used for validation, you should
  point to it here.  If there are no plans for validation, you should state that
  here.}

\wss{You might want to use review sessions with the stakeholder to check that
the requirements document captures the right requirements.  Maybe task based
inspection?}

\wss{For those capstone teams with an external supervisor, the Rev 0 demo should 
be used as an opportunity to validate the requirements.  You should plan on 
demonstrating your project to your supervisor shortly after the scheduled Rev 0 demo.  
The feedback from your supervisor will be very useful for improving your project.}

\wss{For teams without an external supervisor, user testing can serve the same purpose 
as a Rev 0 demo for the supervisor.}

\wss{This section might reference back to the SRS verification section.}

\section{System Tests}

\wss{There should be text between all headings, even if it is just a roadmap of
the contents of the subsections.}

\subsection{Tests for Functional Requirements}

\wss{Subsets of the tests may be in related, so this section is divided into
  different areas.  If there are no identifiable subsets for the tests, this
  level of document structure can be removed.}

\wss{Include a blurb here to explain why the subsections below
  cover the requirements.  References to the SRS would be good here.}

\subsubsection{Area of Testing1}

\wss{It would be nice to have a blurb here to explain why the subsections below
  cover the requirements.  References to the SRS would be good here.  If a section
  covers tests for input constraints, you should reference the data constraints
  table in the SRS.}
		
\paragraph{Title for Test}

\begin{enumerate}

\item{test-id1\\}

Control: Manual versus Automatic
					
Initial State: 
					
Input: 
					
Output: \wss{The expected result for the given inputs.  Output is not how you
are going to return the results of the test.  The output is the expected
result.}

Test Case Derivation: \wss{Justify the expected value given in the Output field}
					
How test will be performed: 
					
\item{test-id2\\}

Control: Manual versus Automatic
					
Initial State: 
					
Input: 
					
Output: \wss{The expected result for the given inputs}

Test Case Derivation: \wss{Justify the expected value given in the Output field}

How test will be performed: 

\end{enumerate}

\subsubsection{Area of Testing2}

...

\subsection{Tests for Nonfunctional Requirements}

\wss{The nonfunctional requirements for accuracy will likely just reference the
  appropriate functional tests from above.  The test cases should mention
  reporting the relative error for these tests.  Not all projects will
  necessarily have nonfunctional requirements related to accuracy.}

\wss{For some nonfunctional tests, you won't be setting a target threshold for
passing the test, but rather describing the experiment you will do to measure
the quality for different inputs.  For instance, you could measure speed versus
the problem size.  The output of the test isn't pass/fail, but rather a summary
table or graph.}

\wss{Tests related to usability could include conducting a usability test and
  survey.  The survey will be in the Appendix.}

\wss{Static tests, review, inspections, and walkthroughs, will not follow the
format for the tests given below.}

\wss{If you introduce static tests in your plan, you need to provide details.
How will they be done?  In cases like code (or document) walkthroughs, who will
be involved? Be specific.}

\subsubsection{Area of Testing1}
		
\paragraph{Title for Test}

\begin{enumerate}

\item{test-id1\\}

Type: Functional, Dynamic, Manual, Static etc.
					
Initial State: 
					
Input/Condition: 
					
Output/Result: 
					
How test will be performed: 
					
\item{test-id2\\}

Type: Functional, Dynamic, Manual, Static etc.
					
Initial State: 
					
Input: 
					
Output: 
					
How test will be performed: 

\end{enumerate}

\subsubsection{Area of Testing2}

...

\subsection{Traceability Between Test Cases and Requirements}

\wss{Provide a table that shows which test cases are supporting which
  requirements.}

\section{Unit Test Description}

\wss{This section should not be filled in until after the MIS (detailed design
  document) has been completed.}

\wss{Reference your MIS (detailed design document) and explain your overall
philosophy for test case selection.}  

\wss{To save space and time, it may be an option to provide less detail in this section.  
For the unit tests you can potentially layout your testing strategy here.  That is, you 
can explain how tests will be selected for each module.  For instance, your test building 
approach could be test cases for each access program, including one test for normal behaviour 
and as many tests as needed for edge cases.  Rather than create the details of the input 
and output here, you could point to the unit testing code.  For this to work, you code 
needs to be well-documented, with meaningful names for all of the tests.}

\subsection{Unit Testing Scope}

\wss{What modules are outside of the scope.  If there are modules that are
  developed by someone else, then you would say here if you aren't planning on
  verifying them.  There may also be modules that are part of your software, but
  have a lower priority for verification than others.  If this is the case,
  explain your rationale for the ranking of module importance.}

\subsection{Tests for Functional Requirements}

\wss{Most of the verification will be through automated unit testing.  If
  appropriate specific modules can be verified by a non-testing based
  technique.  That can also be documented in this section.}

\subsubsection{Module 1}

\wss{Include a blurb here to explain why the subsections below cover the module.
  References to the MIS would be good.  You will want tests from a black box
  perspective and from a white box perspective.  Explain to the reader how the
  tests were selected.}

\begin{enumerate}

\item{test-id1\\}

Type: \wss{Functional, Dynamic, Manual, Automatic, Static etc. Most will
  be automatic}
					
Initial State: 
					
Input: 
					
Output: \wss{The expected result for the given inputs}

Test Case Derivation: \wss{Justify the expected value given in the Output field}

How test will be performed: 
					
\item{test-id2\\}

Type: \wss{Functional, Dynamic, Manual, Automatic, Static etc. Most will
  be automatic}
					
Initial State: 
					
Input: 
					
Output: \wss{The expected result for the given inputs}

Test Case Derivation: \wss{Justify the expected value given in the Output field}

How test will be performed: 

\item{...\\}
    
\end{enumerate}

\subsubsection{Module 2}

...

\subsection{Tests for Nonfunctional Requirements}

\wss{If there is a module that needs to be independently assessed for
  performance, those test cases can go here.  In some projects, planning for
  nonfunctional tests of units will not be that relevant.}

\wss{These tests may involve collecting performance data from previously
  mentioned functional tests.}

\subsubsection{Module ?}
		
\begin{enumerate}

\item{test-id1\\}

Type: \wss{Functional, Dynamic, Manual, Automatic, Static etc. Most will
  be automatic}
					
Initial State: 
					
Input/Condition: 
					
Output/Result: 
					
How test will be performed: 
					
\item{test-id2\\}

Type: Functional, Dynamic, Manual, Static etc.
					
Initial State: 
					
Input: 
					
Output: 
					
How test will be performed: 

\end{enumerate}

\subsubsection{Module ?}

...

\subsection{Traceability Between Test Cases and Modules}

\wss{Provide evidence that all of the modules have been considered.}
				
\bibliographystyle{plainnat}

\bibliography{../../refs/References}

\newpage

\section{Appendix}

This is where you can place additional information.

\subsection{Symbolic Parameters}

The definition of the test cases will call for SYMBOLIC\_CONSTANTS.
Their values are defined in this section for easy maintenance.

\subsection{Usability Survey Questions?}

\wss{This is a section that would be appropriate for some projects.}

\end{document}